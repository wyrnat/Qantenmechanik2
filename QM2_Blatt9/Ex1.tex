\section{Dynamics of a driven two-level system}

\begin{align}
H \left( t \right)=
\begin{pmatrix}
E_1 & 0 \\ 0 & E_2
\end{pmatrix}
+
\begin{pmatrix}
0 & \gamma e^{i \omega t} \\ \gamma e^{-i \omega t} & 0
\end{pmatrix}
\end{align}
Dieser zeitabh\"angige Hamiltonien mit kinetischem Term $H_0$ und Austauschterm $V_{12}$ beschreibt ein 2-Levelsystem.

\subsection{a)}

Die Wellenfunktion wird als Fockzustand angesetzt

\begin{align}
\Ket{\Psi} = \sum_n c_n \Ket{n} = c_1 \Ket{1} + c_2 \Ket{2}
\end{align}

Im \textbf{Interaction picture} wird eine Wellenfunktion beschrieben als $\Ket{Psi_I} = e^{\frac{i}{\hbar}H_0 t} \Ket{\Psi_s}$.\\
Nun bringen wir den Hamiltonien in das  \textbf{Interaction picture}.

\begin{align}
H_I = e^{\frac{i}{\hbar}H_0 t} H e^{-\frac{i}{\hbar}H_0 t} \\
= H_0 + V_I
\end{align}

Da $H_0$ mit dem Exponentialterm kommutiert, ist er in beiden Methoden identisch.\\
Nun benutzen wir die zeitabh\"angige Schr\"odinger-Gleichung, um die Besetzungswahrscheinlichkeiten zu ermitteln. Daf\"ur wird von der linken Seite ein $\Bra{n}$ 

\begin{align}
i \hbar \partial_t c_n = \sum_m \Bra{n} V_I \Ket{m} c_m
\end{align}
Bestimmung der Terme
\begin{align}
\Bra{1} V_I \Ket{2} = 
\begin{pmatrix}
1 & 0
\end{pmatrix}
e^{\frac{i}{\hbar}H_0 t}
\begin{pmatrix}
0 & \gamma e^{i \omega t} \\ \gamma e^{-i \omega t} & 0
\end{pmatrix}
e^{-\frac{i}{\hbar}H_0 t}
\begin{pmatrix}
0 \\ 1
\end{pmatrix}
\\
= e^{\frac{i}{\hbar}E_1 t} e^{-\frac{i}{\hbar}E_2 t}
\begin{pmatrix}
1 & 0
\end{pmatrix}
\begin{pmatrix}
\gamma e^{-i \omega t} \\ 0
\end{pmatrix}
\\
= \gamma e^{-i \omega t} e^{\frac{i}{\hbar}\left(E_1-E_2\right) t}
\end{align}
Selbiges gilt f\"ur $\Bra{2} V_I \Ket{1}$, aber mit Minuszeichen in den Exponenten der e-Funktion. F\"ur $\Bra{1} V_I \Ket{1}$ und $\Bra{2} V_I \Ket{2}$ ergibt sich aufgrund einer reinen Off-Diagonal-Matrix \textbf{0}.\\
Damit erhalten wir zwei Gleichungen f\"ur die Besetzungswahrscheinlichkeiten:

\begin{align}
i \hbar \partial_t c_1 = \gamma e^{i \phi t} c_2 \\
i \hbar \partial_t c_2 = \gamma e^{-i \phi t} c_1
\end{align}
Mit $\phi = \omega - \frac{E2-E1}{\hbar}$. Nun entkoppeln wir die Gleichungen, dazu wird die Ableitung der Gleichung zur Hilfe genommen

\begin{align}
i \hbar \partial_t^2 c_1 = \gamma e^{i \phi t} \partial_t c_2 + i \phi \gamma e^{i \phi t}  c_2 \\
= \gamma e^{i \phi t} \left( i \phi \frac{i \hbar \partial_t c_1}{\gamma e^{i \phi t}} + \frac{\gamma e^{-i \phi t} c_1}{i \hbar} \right) \\
= - \hbar \phi \partial_t c_1 - \frac{i \gamma^2}{\hbar} c_1
\end{align}
Selbiges Verfahren f\"ur $c_2$. Damit ergeben sich zwei separierte Differentialgleichungen

\begin{align}
\partial_t^2 c_1 = i \phi \partial_t c_1 - \frac{\gamma^2}{\hbar^2} c_1 \\
\partial_t^2 c_1 = -i \phi \partial_t c_1 - \frac{\gamma^2}{\hbar^2} c_1
\end{align}
Wir w\"ahlen einen Exponentialansatz $e^{i \Gamma t}$ und erhalten (f\"ur $c_1$)
\begin{equation}
\Gamma_1 = \frac{\phi}{2} \pm \frac{1}{2} \sqrt{\frac{4 \gamma^2}{\hbar^2} + \phi^2}
\end{equation}
mit den Anfangsbedingungen
\begin{align}
c_1 \left( 0 \right) = 1 \\
c_2 \left( 0 \right) = 0 \\
\dot{c}_1 \left( 0 \right) = 0 \\
\dot{c}_2 \left( 0 \right) = \frac{\gamma}{i \hbar} \\
\end{align}
Der Term unter der Wurzel wird nun als $\Omega$ bezeichnet.\\
Damit folgt f\"ur die Wahrscheinlichkeiten

\begin{align}
c_1 \left( t \right) = \frac{e^{i \frac{\theta}{2}t}}{\Omega} \left[ \Omega cos \left( \frac{\Omega}{2} t \right) - \phi i sin \left( \frac{\Omega}{2}t \right) \right] \\
|c_1|^2 = 1 - \frac{4 \gamma^2}{\hbar^2 \Omega^2} sin^2 \left( \frac{\Omega}{2} t \right) \\
|c_2|^2 = \frac{4 \gamma^2}{\hbar^2 \Omega^2} sin^2 \left( \frac{\Omega}{2} t \right) \\
\end{align}

\subsection{b)}

\begin{align}
max \left( |c_2|^2 \right) = \frac{4 \gamma^2}{\hbar^2 \Omega^2} \\
= \frac{\gamma}{\gamma + \frac{\hbar^2 \phi^2}{4}} \leq 1
\end{align}
Die Besetzungswahrscheinlichkeiten oszillieren um $180^{\circ}$ phasenverschoben. Wenn $\phi = 0$, dann ist es m\"oglich, das jeweils ein Niveau komplett entv\"olkert ist. Diese \"Anderung der Besetzung nennt man auch \textbf{Rabi-Oszillation}.
