\section{Scattering of a hard-sphere potential}

Ziel ist die Bestimmung der Phasenverschiebung und der Streuquerschnitt. Das
Potential lautet
\begin{align}
V \left(r\right) = 
\begin{Bmatrix}
\infty, & r<R \\
0, & r>R
\end{Bmatrix}
\end{align}

\subsection{a)}

Randbedingung bei $r=R$

\begin{align}
0 = A-l \left(R\right) \\
= e^{i\delta_l} \left[\gamma_l\left(R\right)cos \left(\delta_l\right) - n_l
\left(R\right) sin \left(\delta_r\right) \right]
\end{align}

mit $tan \left(\delta_l\right) = \frac{\gamma_l \left(\kappa R\right)}{n_l
\left(\kappa R\right)} $

\subsection{b)}

\begin{align}
tan \left(\delta_0\right) = \frac{\gamma_0}{n_0} \\
= \frac{\frac{1}{kR} sin \left(kR\right)}{-\frac{1}{kR} cos\left(kR\right) } =
-tan \left(kR\right)
\end{align}

$\delta_0$ betrifft die ``s-wave'' Streuung mit Drehimpuls $l=0$. Die
Wellenfunktion lautet dann
	
\begin{align}
A_0 \left(r\right) = e^{ikR} \left[ \frac{\sin{kR}}{kR} \cos{kR} -
\frac{\cos{kr}}{kr} \sin{kR} \right] \\
= e^{ikR} \left[ \frac{sin \left(kr-kR\right)}{kR} \right]
\end{align}

\subsection{c)}

Mit $k \rightarrow 0$:

\begin{align}
\tan{\delta_l} = \frac{\gamma_l\left(kR\right) }{n_l\left(kR\right) } \\
\rightarrow
\frac{-\left(kR\right)^{2l+1}}{\left(2l+1\right)!!\left(2l-1\right)!!}
=
\frac{-\left(kR\right)^{2l+1}}{\left(2l+1\right)\left[\left(2l-1\right)!!\right]^2}
\end{align}

Nun wird gezeigt, dass nur die Ber\"ucksichtigung von ``s-wave'' Streuung bei
kleinen Energien ausreichend ist.\\
Mit $k \rightarrow 0$:

\begin{align}
\frac{\delta_1}{\delta_0} =
\frac{\arctan{\frac{\left(kR\right)^3}{3}}}{\arctan{kR}} =
\frac{\frac{\left(kR\right)^3}{3}}{kR-\frac{\left(kR\right)^3}{3}} \\
= \frac{1}{\frac{3}{\left(kR\right)^2}-1} \approx \frac{\left(kR\right)^2}{3}
\rightarrow 0
\end{align}

Allgemein gilt

\begin{align}
\frac{\delta_{l+1}}{\delta_l} \propto \frac{c_{l+1}}{c_l} \left(kR\right)^2
\end{align}

Mit $c_l = \frac{1}{\left(2l+1\right)\left[\left(2l-1\right)!! \right]^2}$

\subsection{d)}

\begin{align}
\sigma_{tot} = \frac{4\pi}{k^2} \sum_{l=0}^{\infty} \left(2l+1\right) sin^2
\left(\delta_l\right) \\
\approx \frac{4\pi}{k^2} sin^2 \left(\delta_0\right) \approx \frac{4\pi}{k^2}
\left(kR\right)^2 \\
= 4\pi R^2
\end{align}
