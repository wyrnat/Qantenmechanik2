\section{Scattering resonances and bound states: scattering off a delta-shell
potential}

\begin{equation}
V \left(r\right) = g \delta\left(r-R\right)
\end{equation}

Wir fokussieren uns auf die ``s-wave'' Streuung, mit
$u\left(r\right)=rA_0\left(r\right)$.

\begin{align}
U \left(r\right) =
\begin{Bmatrix}
U_1 \left(r\right) = e^{ikr} - \left(1+2ikf_0\left(k\right) \right) e^{ikr} &
,r>R \\
U_2 \left(r\right) C \; sin \left(kr\right) & ,r<R
\end{Bmatrix}
\end{align}

\subsection{a)}
 Bestimmung des Wertes am Ort R
 
 \begin{align}
 0 = \int_{R-\epsilon}^{R+\epsilon} dr \; \left[ E + \frac{\hbar^2}{2M}
 \partial_r^2 - g \delta \left(r-R\right) \right] U \left(r\right) \\
 E \int_{r-\epsilon}^{R+\epsilon} dr \; U \left(r\right) + \frac{\hbar^2}{2M}
 \left[ \partial_r U |_{R+\epsilon} - \partial_r U |_{R-\epsilon} \right] - g
 U\left(R\right) \\
 = 0 + U'_1 \left(R\right) - U'_2 \left(R\right) - g U \left(R\right)
 \end{align}
 
 Es gelten folgende Randbedingungen
 
 \begin{align}
 \left(1\right) \; \; U_1 \left(R\right) = U_2 \left(R\right) \\
 \left(2\right) \; \;  U'_1 \left(R\right) = U'_2 \left(R\right) +
 \frac{2M}{\hbar^2} g U \left(R\right)
 \end{align}
 
 \subsection{b)}
 
 Mit den Randbedingungen lassen sich Potential und Streuamplitude bestimmen
 
 \textbf{(2):}
 
 \begin{align}
 Ck \; cos \left(kR\right) + \frac{2Mg}{\hbar^2} c \; sin \left(kR\right) = -ik
 \left( e^{ikR} + \left( 1 + 2ik f_0 \left(k\right) \right) e^{ikR} \right) \\
 \end{align}
 
 \textbf{(1):}
 
 \begin{align}
 C \; sin \left(kR\right) + 2\frac{Mg}{\hbar^2} = ik \left( \frac{e^{ikr} +
 \left(1+2ikf_0\left(k\right) \right) e^{ikR} }{e^{ikr} -
 \left(1+2ikf_0\left(k\right) \right) e^{ikR}} \right)
 \end{align}
 
 Zusammengefasst ergibt sich damit
 
 \begin{align}
 1 + 2ikf_0 \left(R\right) = e^{i2\delta_0} = \frac{cot\left(kR\right) + u + i
 }{cot\left(kR\right) + u - i} e^{-2ikR}
 \end{align}
 
 mit
 
 \begin{align}
 u = \frac{2Mg}{\hbar^2k} \\
 f_0 \left(k\right) = e^{i\delta_0} \frac{sin \left(\delta_0\right)}{k}
 \end{align}
 
 \subsection{c)}
 
 Es gilt
 
 \begin{align}
 arg \left(x+iy\right) = \arctan{\left(\frac{y}{x}\right)} \\
 arg \left(\frac{a}{l} \right) = arg \left(a\right) - arg \left(l\right)
 \end{align}
 
 \begin{align}
 \gamma = \cot{\left(kR\right)} + \mu \\
 und \\
 e^{i2\delta_0} = \frac{\gamma +i}{\gamma -i} e^{-2ikR}
 \end{align}
 
 Nun formen wir den Bruch um
 
 \begin{align}
 arg \left( \frac{\gamm +i}{\gamma -i} \right) = arg \left(\gamma +i\right) -
 arg \left(\gamma -i\right) \\
 = \arctan{\left(\frac{1}{\gamma} \right)} - \arctan{\left(-\frac{1}{\gamma}
 \right)} \\
 = 2 \arctan{\frac{1}{\gamma} }
 \end{align}
 
 Damit ergibt sich
 
 \begin{align}
 2 \gamma_0 = 2 \arctan{\frac{1}{\gamma} } - 2kR \\
 \gamma_0 = \arctan{\left( \frac{1}{\cot{\left(kR\right)} + \mu } \right) } - kR
 \end{align}
 
 \subsection{d)}
 
 Ein sp\"arischer Potentialtopf unendlicher H\"ohe hat als m\"ogliche L\"osung
 \\
 $u\left(r\right) \approx \sin{kR} $. \\
 Die ``cos''-L\"osung kann aufgrund der Randbedingungen ausgeschlossen w.erden
 
 