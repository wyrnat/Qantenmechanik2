\section{The CC math}

Der \"Ubergangsoperator f\"ur Anregungen ist $T_{\mu} = a_{ru}^{\dagger} a_{bi} a_{ql} ...$

\subsection{a)}

Zu erwarten ist: $T_{\mu}T_{\mu} = 0$\\
\begin{align}
T_{\mu}T_{\mu} = \Pi_{j=1}^{m} a_{rn_j}^{\dagger} a_{bq_j} \; \Pi_{i=1}^{m} a_{rn_i}^{\dagger} a_{bq_i} \\
= \Pi_{j=1}^{m-1} a_{rn_j}^{\dagger} a_{bq_j} \left( a_{rn_m}^{\dagger} a_{bq_m} \right) a_{rn_1}^{\dagger} a_{bq_1} \Pi_{i=2}^{m} ... \\
= \Pi_{j=1}^{m-1} a_{rn_j}^{\dagger} a_{bq_j} a_{rn_m}^{\dagger} \Pi_{i=1}^{m-1} a_{rn_i}^{\dagger} a_{bq_i} a_{nk_m}^{\dagger} \; a_{bq_m} a_{bq_m} \\
= \hat{O} a_{bq_m} a_{bq_m} = 0
\end{align}

\subsection{b)}

Wurde nicht gemacht, zu beweisen ist $\left[ \tau_{\mu} , \tau_{\nu} \right]$

\subsection{c)}

i)
\begin{align}
e^{t_{\mu} \tau_{\mu}} = \sum_{n=0}^{\inf} \frac{\left(t_{\mu} \tau_{\mu} \right)^n}{n!} \\
= 1 \; + \; t_{\mu} \tau_{\mu} \; + \; \tau_{\mu} \tau_{\mu} \sum_{n=2}^{\inf} \frac{\left(t_{\mu} \tau_{\mu} \right)^n}{n!} \\
= 1 \; + \; t_{\mu} \tau_{\mu}
\end{align}
Da das Produkt $\tau_{\mu} \tau_{\mu}$ stets 0 ist.\\
\\
ii)
\begin{equation}
\Pi_{\mu} \left(1 + t_{\mu} \tau_{\mu} \right) = \Pi_{\mu} e^{t_{\mu} \tau_{\mu}} =^* e^{\sum_{\mu} t_{\mu} \tau_{\mu}} = e^{\hat{T}}
\end{equation}
Bei * wird Die Baker-Campell-Haussdorf Relation und das Ergebnis von b) verwendet.

Alles in allem folgt daraus: $\Ket{CC} =  e^{\hat{T}} \Ket{\Psi_0}$

\subsection{d)}

\begin{equation}
e^{\hat{T}} \Ket{\Psi_0} = \sum_{i=0}^N \hat{C}_i \Ket{\Psi_0}
\end{equation}
Die Reihenentwicklung von $e^{\hat{T}}$ lautet
\begin{equation}
e^{\hat{T}} = 1 + \hat{T} + \frac{1}{2!}\hat{T}^2 + \frac{1}{3!}\hat{T}^3 + \frac{1}{4!}\hat{T}^4 + ...
\end{equation}
T ist nun eine Summe, dessen Terme wir als $T_n$ auffassen [Siehe c)].
\begin{align}
T = T_1 + T_2 + T_3 + T_4 + ... \\
T^2 = T_1^2 + 2T_1T_2 + 2T_1T_3 + T_2^2 + ... \\
T^3 = T_1^3 + 3T_1^2T_2 + ... \\
T^4 = T_1^4 + ...
\end{align}
Damit folgt (Verstehe ich nicht???)
\begin{equation}
e^{\hat{T}} = 1 + T_1 + \left[ T_2 + \frac{1}{2} T_1^2 \right] + \left[ T_3 + T_1T_2 + \frac{1}{6} T_1^3 \right]
+ \left[ T_4 + T_1T_3 + \frac{1}{2} T_1^2T_2 +\frac{1}{24}T_1^4 \right]
\end{equation}
