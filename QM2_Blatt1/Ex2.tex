\section{Nonlinear oscillator in 1D}

%------------------------------------------------

Ansatz:
\begin{align}
\Ket{\Phi} = c_{1} \Ket{\Psi_{0}} + c_{2} \Ket{\Psi_{2}} \\
\Ket{Psi_{0}} = \left( \frac{\omega}{\pi} \right)^{\frac{1}{4}} e^{-\frac{1}{2}\omega x^2} \\
\Ket{\Psi_{2}} = \frac{1}{\sqrt{2}} \left( \frac{\omega}{\pi} \right)^{\frac{1}{4}} e^{-\frac{1}{2}\omega x^2} \left( 2\omega x^2 -1 \right)
\end{align}

Nun bestimmen wir die Matrixelemente $\Bra{\Psi_{i}} H \Ket{\Psi_{j}}$, um danach per Determinanate die Eigenwerte des Hamiltoniens zu ermitteln. Das sind die Energien des Systems mit den beiden Funktionen.

\begin{align}
\Bra{\Psi_{0}} H \Ket{\Psi_{0}} = \frac{\omega}{2} + \frac{3}{16}  \frac{\lambda}{\omega^2} \\
\Bra{\Psi_{2}} H \Ket{\Psi_{2}} = \frac{5}{2} \omega + \frac{39}{16} \frac{\lambda}{\omega^2} \\
\Bra{\Psi_{0}} H \Ket{\Psi_{2}} = \Bra{\Psi_{2}} H \Ket{\Psi_{0}} = \frac{1}{\sqrt{2}} \frac{3}{4} \frac{\lambda}{\omega^2}
\end{align}
Damit folgt f\"ur die Energie mit $det \left( H - E\mathbbm{1} \right) $:
\begin{align}
E_{0}^{mess} = 0,5596 > E_{0} \\
E_{2}^{mess} = 3,4904
\end{align}


