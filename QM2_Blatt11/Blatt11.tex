%%%%%%%%%%%%%%%%%%%%%%%%%%%%%%%%%%%%%%%%%
% Short Sectioned Assignment
% LaTeX Template
% Version 1.0 (5/5/12)
%
% This template has been downloaded from:
% http://www.LaTeXTemplates.com
%
% Original author:
% Frits Wenneker (http://www.howtotex.com)
%
% License:
% CC BY-NC-SA 3.0 (http://creativecommons.org/licenses/by-nc-sa/3.0/)
%
%%%%%%%%%%%%%%%%%%%%%%%%%%%%%%%%%%%%%%%%%

%----------------------------------------------------------------------------------------
%	PACKAGES AND OTHER DOCUMENT CONFIGURATIONS
%----------------------------------------------------------------------------------------

\documentclass[paper=a4, fontsize=11pt]{scrartcl} % A4 paper and 11pt font size

\usepackage[utf8]{inputenc}
\usepackage[ngerman]{babel} % English language/hyphenation
\usepackage{amsmath,amsfonts,amsthm} % Math packages
\usepackage{braket}
\usepackage{bbm}

\usepackage{lipsum} % Used for inserting dummy 'Lorem ipsum' text into the template

\usepackage{sectsty} % Allows customizing section commands
\allsectionsfont{\centering \normalfont\scshape} % Make all sections centered, the default font and small caps


\numberwithin{equation}{section} % Number equations within sections (i.e. 1.1, 1.2, 2.1, 2.2 instead of 1, 2, 3, 4)
\numberwithin{figure}{section} % Number figures within sections (i.e. 1.1, 1.2, 2.1, 2.2 instead of 1, 2, 3, 4)
\numberwithin{table}{section} % Number tables within sections (i.e. 1.1, 1.2, 2.1, 2.2 instead of 1, 2, 3, 4)

\setlength\parindent{0pt} % Removes all indentation from paragraphs - comment this line for an assignment with lots of text

%----------------------------------------------------------------------------------------
%	TITLE SECTION
%----------------------------------------------------------------------------------------


\title{	
\normalfont \normalsize 
\textsc{Quantenmechanik II, Prof. Peter Schmelcher} \\ [25pt] % Your university, school and/or department name(s)
}

\author{Blatt 11: L\"osungen} % Your name

\date{\normalsize\today} % Today's date or a custom date

\begin{document}

\maketitle % Print the title

%----------------------------------------------------------------------------------------
%	PROBLEMS
%----------------------------------------------------------------------------------------

\section{Born approximation}

\subsection{a)}

\begin{align}
f^{\left(1\right)} \left(\vec{q}\right) = - \frac{1}{4\pi} \frac{2M}{\hbar^2}
\int d^3r \; e^{iqr} V\left(\vec{r}\right) \\
= - \frac{2M}{\hbar^2q} \int_0^{\infty} dr \; rV\left(r\right)
sin\left(qr\right)
\end{align}

Mit einem radialen Potential \\
$V\left( \vec{r} \right)= V\left(r\right) = V_0
\frac{e^{i\kappa r}}{r}$ \\

verschwindet die lineare Abh\"angigkeit des
Integrals.

\begin{align}
= - \frac{2M}{\hbar^2 q} \frac{1}{2r} \int_0^{\infty} dr \; \left[ e^{i\kappa r
+iqr}-e^{-\kappa r -iqr} \right] \\
= \frac{2m V_0}{\hbar^2q} \frac{1}{2i} \left[
-\frac{1}{\kappa+iq}-\frac{-1}{\kappa-iq} \right] \\
= - \frac{2MV_0}{\hbar^2\left(q^2+\kappa^2\right)} \\
\lim\limits{\kappa \rightarrow 0} \left(\ldots\right) = - \frac{2MV_0}{\hbar^2
q^2}
\end{align}

\subsection{b)}

\begin{align}
V\left(\vec{r}\right) = \int dr' \; \zeta \left(\vec{r}'\right)
W\left(\vec{r}-\vec{r}'\right) \\
f^{\left(1\right)} \left(\vec{q}\right) = - \frac{1}{4\pi} \frac{2M}{\hbar^2}
\int dr \; e^{iq\vec{r}} \int dr' \; \zeta \left(\vec{r}'\right)
W\left(\vec{r}-\vec{r}'\right) \\
= -\frac{1}{e\pi} \frac{2M}{\hbar^2} \int d\eta \; e^{iq
\left(\vec{\eta}+\vec{r}' \right)} \int dr' \; \zeta \left(\vec{r}'\right)
W\left(\vec{\eta}\right) \\
= -\frac{1}{4\pi} \frac{2M}{\hbar^2} \left[ \int d\eta \; e^{iq\vec{\eta}}
W\left(\vec{\eta}\right) \right] \left[ \int dr' \; \zeta \left(\vec{r}'\right)
e^{iq\vec{r}'} \right] \\
\\
= f_W^{\left(1\right)} \left(\vec{q}\right) \tilde{\zeta}
\left(\vec{q}\right)
\end{align}

Mit $\eta = \vec{r}-\vec{r}'$.\\
In der Scattering Funktion $f$ steckt somit der Formfaktor der Probe mit drin.

\section{Dirac $\delta$ identities}

\subsection{a)}

\begin{align}
\delta_{\epsilon} \left(x\right) := \frac{1}{\pi} Im \left(
\frac{1}{x+i\epsilon} \right) \\
= - \frac{1}{2\pi i} \left[ \frac{1}{x+ i\epsilon} - \frac{1}{x- i\epsilon}
\right] \\
= - \frac{1}{2 \pi i} \left[ - \frac{2 i \epsilon}{x^2 + \epsilon^2} \right] \\
= \frac{1}{\pi} \frac{\epsilon}{x^2 + \epsilon^2}
\end{align}

\subsection{b)}

\begin{align}
\delta \left( F\left(x\right) \right) = \sum_j \frac{1}{|F' \left(x_j\right)|}
\delta \left(x-x_j\right)
\end{align}
Mit $F\left(x_j\right)=0$ und $F'\left(x_j\right) \neq 0$. Das zu l\"osende
Integral behinhaltet nun die Faltung der $\delta$-Funktion mit einer
testfunktion, wir wollen die Funktion am Punkt $x_0$ auswerten.

\begin{align}
I = \int_{x_0-\epsilon}^{x_0+\epsilon} dx \; \delta \left( F\left(x\right)
\right) \Phi \left(x\right) \\
= \int_{F\left(x_0-\epsilon\right)}^{F\left(x_0+\epsilon\right)} dx \;
\frac{1}{|F' \left(x\left(y\right) \right)|} \delta \left(x\right) \Phi
\left(x\left(y\right) \right)
\end{align}
Nun nehmen wir an: $y = F \left(x\right)$.
Wir invertieren den Ausdruck nach $x\left(y\right)$ mit $x\left(0\right)=x_0$
und k\"onnen zwischen 2 F\"allen unterscheiden:\\
\\
\textbf{1. Fall}: $F'\left(x_0\right) > 0$
\begin{align}
\Rightarrow F \left(x_0 + \epsilon \right) > F \left(x_0 - \epsilon \right) \\
I = frac{1}{F' \left(x_0 \right)} \Phi \left(x_0\right) = frac{1}{|F' \left(x_0
\right)|} \Phi \left(x_0\right)
\end{align}

\textbf{2. Fall}: $F'\left(x_0\right) < 0$
\begin{align}
\Rightarrow F \left(x_0 + \epsilon \right) < F \left(x_0 - \epsilon \right) \\
I = \int_{F\left(x_0-\epsilon\right)}^{F\left(x_0+\epsilon\right)} dx \;
\frac{1}{|F' \left(x\left(y\right) \right)|} \delta \left(x\right) \Phi
\left(x\left(y\right) \right) \\
= frac{1}{|F' \left(x_0\right)|} \Phi \left(x_0\right)
\end{align}

Von beiden Seiten kommen wir zum gleichen Ergebnis. Der Wert am Punkt $x_0$ ist
damit eindeutig bestimmt.



\section{Kratzer's potential for diatomic molecules}

\subsection{a)}

\begin{align}
V \left(\rho\right) = -2D \left(\frac{1}{\rho} - \frac{1}{2\rho^2} \right)
\end{align}

Nun nehmen wir an, dass die L\"osungen die Form \\
$U \left(r,\theta, \phi\right) = \frac{C}{\rho} X_l \left(\rho\right) Y_{lm}
\left(\theta, \phi\right)$ \\

F\"ur den radialen Teil gilt dann:

\begin{align}
\frac{\partial^2 X_l}{\partial \rho^2} + \left[-b^2 +
\frac{2\gamma}{\rho} - \frac{\gamma^2 + l \left(l+1\right)}{\rho^2} \right] X_l
\left(\rho\right) = 0
\end{align}

\subsection{b}



%----------------------------------------------------------------------------------------



\end{document}