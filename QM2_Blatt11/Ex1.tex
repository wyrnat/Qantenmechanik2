\section{The Residue Theorem}

In dieser Aufgabe behandeln wir das Lippmann-Schwinger Theorem und die
Behandlung von Integralen mit dem Residuensatz. Der Residuensatz lautet:
\begin{align}
\oint_c f \left( z \right) dz = 2 \pi i \sum_{k=1}^{N} Res \left(f,a_k\right)
\end{align}
mit $a_k$ als den Polstellen von $f\left(z\right)$ und der Residue
\begin{align}
Res \left(f, a_k\right) = \frac{1}{\left(n-1\right)!} \lim \limits_{z
\rightarrow a_k} \frac{\partial^{n-1}}{\partial z^{n-1}} \left[
\left(z-a_k\right)^n f \left(z\right) \right]
\end{align}

\subsection{a)}

Zu zeigen: $\int_0^{\infty} \frac{dx}{\left(x^2+1\right)^2} = - \frac{\pi}{4} $

\begin{align}
\int_{-R}^R \frac{dx}{\left(x^2+1\right)^2} = \oint_C
\frac{dz}{\left(z^2+1\right)^2} - \int_S
\frac{dz}{\left(z^2+1\right)^2}
\end{align}

Nach dem Residuensatz integriert man von $- \infty$ nach $\infty$ und schlie\"st
das Integral zu einem geschlossenen Wegintegral, indem man imagin\"ar \"uber
einen Halbkreis im 2dim Raum integriert. Das Integral entlang $C$ beschreibt
diesen Weg, das Integral $S$ zieht den Halbkreis wieder ab. \\
$S$ wird beschrieben mit $S = R e^{i \alpha}$. F\"ur $R \rightarrow \infty$
ergibt sich:
\begin{equation}
\int_S \frac{dz}{\left(z^2+1\right)^2} = 0
\end{equation}
Siehe Blatt 10.\\
Nun werden wir das Integral mithilfe des Residuensatzes in 3 Schritten l\"osen:
\\
\textbf{i)} Finde die Polstellen $a_f$

\begin{align}
\left(z^2+1\right)^2 = \left(z-i\right)^2 \left(z+i\right)^2 \\
\Rightarrow \; a_1 = i \\
\Rightarrow \; a_2 = -i
\end{align}

\textbf{ii)} Bestimmung der Residuen \\
Nun bestimmen wir die Residue f\"ur $a_2$, da nur diese innerhalb des
geschlossenen Wegintegrals liegt. $a_1$ wird nicht eingeschlossen und spielt als
Polstelle deshalb keine Rolle

\begin{align}
Res \left( \frac{1}{\left(z^2+1\right)^2} \; , \; i \right) =
\frac{\partial}{\partial z} \left[ \left(z-i\right)^2
\frac{1}{\left(z-i\right)^2 \left(z+i\right)^2} \right] \\
= \frac{\partial}{\partial z} \frac{1}{\left(z+i\right)^2} \\
= \frac{-2}{\left(z+i\right)^3} \bigg|_{z=i} = \frac{1}{4i}
\end{align}
Somit l\"asst sich das Integral l\"osen

\begin{align}
\int_0^{\infty} \frac{dx}{\left(x^2+1\right)^2} = \frac{1}{2} \lim \limits_{R
\rightarrow \infty} \int_{-R}^{R} \frac{dx}{\left(x^2+1\right)^2} \\
= \frac{1}{2} \lim \limits_{R \rightarrow \infty} \oint_C
\frac{dz}{\left(z^2+1\right)^2} \\
= \frac{1}{2} 2 \pi i \left( \frac{1}{4i} \right) \\
= \frac{\pi}{4}
\end{align}

\subsection{b)}
Zu zeigen:

\begin{align}
\int_{-\infty}^{\infty} dq \frac{q \left(e^{iqx}-e^{-iqx}\right)}{q^2 -k^2 \mp
i\epsilon} = 2  \pi i e^{\pm ikx}
\end{align}

\begin{align}
\int_{-\infty}^{\infty} dq \frac{q \left(e^{iqx}-e^{-iqx}\right)}{q^2 -k^2 \mp
i\epsilon} \\
= 2 \int_{-\infty}^{\infty} dq \frac{q e^{iqx}}{q^2 -k^2 \mp
i\epsilon} \\
= 2 \int_{-\infty}^{\infty} dq \; f \left(q\right)
\end{align}

\textbf{i)} Bestimmung der Polstellen

\begin{align}
q^+ = h \sqrt{1 \pm \frac{i\epsilon}{k}} \approx h \pm \epsilon ' \\
q^- = -h \sqrt{1 \pm \frac{i\epsilon}{k}} \approx -h \mp \epsilon '
\end{align}
Mit $\epsilon ' = \frac{\epsilon}{2k}$. \\
\\
\textbf{ii)} Bestimmung der Residuen \\

\begin{align}
Res \left( f \left(q\right) , q^{\pm} \right) = \left( q - q^{\pm} \right)
\frac{q e^{iqx}}{\left(q-q^-\right)\left(q-q^+\right)} \bigg|_{q=q^{\pm}} \\
= \frac{q e^{iqx}}{\left(q-q^+\right)} \bigg|_{q=q^{\pm}} \\
= \frac{q^{\pm}e^{iq^{\pm}x}}{2q^{\pm}} \\
= \frac{1}{2} e^{\pm ikx} e^{\mp \epsilon x}
\end{align}
F\"ur $\epsilon \rightarrow 0$ bedeutet dies

\begin{equation}
Res \left( f \left(q\right) , q^{\pm} \right) = \frac{1}{2} e^{\pm ikx}
\end{equation}
Damit k\"onne wir nun das Integral l\"osen

\begin{align}
2 \int_{-\infty}^{\infty} dq \; f \left(q\right) = 2 \pi i Res \left(f
\left(q\right) \right) \\
= 2 \pi i e^{\pm ikx}
\end{align}
